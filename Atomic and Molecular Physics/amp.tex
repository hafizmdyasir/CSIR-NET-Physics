% amp.tex
% Created by Mohammad Yasir.


\documentclass[12pt]{report}

\usepackage{parskip, amsmath, amssymb, titling}
\usepackage[a4paper, top=1.00in, bottom=1.00in, left=1.00in, right=1.00in]{geometry}

\usepackage[T1]{fontenc}
\usepackage{mlmodern}

\usepackage{../styles}
\pagestyle{plain}

\title{Atomic and Molecular Physics}
\author{Mohammad Yasir}
\date{\today}

\begin{document}
\begin{Huge}
    \thetitle
\end{Huge}\\
\begin{small}
    \textbf{Compiled by :} Mohammad Yasir\\
    \textbf{Updated \hspace{1.78em}:} \thedate
\end{small}
\hrule

% PART B
    \section*{Part B - Advanced}
    \textbf{Quick Note:} Atomic and Molecular Physics is not included in the Core Part A syllabus.


    \begin{entry}
        \question{Diffuse hydrogen gas within a galaxy may be assumed to follow a Maxwell distribution at temperature $10^6\ K$, while the temperature appropriate for the H gas in the inter-galactic space, following the same distribution, may be taken to be $10^4\ K$. The ratio of thermal broadening of the Lyman-$ \alpha$ line from the H-atoms within the galaxy to that from the inter-galactic space is closest to. \\ (February 15, 2022)}
        \options{
            \option{A} & 100 & \option{B} & 1/100\\
            \option{C} & 10  & \option{D} & 1/10\\
        }
    \end{entry}

    \begin{entry}
        \question{The absorption lines arising from pure rotational effects of HCl are observed at $83.03\ cm^{-1}$, $103.73\ cm^{-1}$, $124.30\ cm^{-1}$, $145.03\ cm^{-1}$ and $165.51\ cm^{-1}$. The moment of inertia of the HCl molecule is
        (take $ \frac{\hbar}{2\pi c} = 5.6 \times 10^{-44} kg.m$)\\ (November 19, 2020)}
        \options{
            \option{A} & $1.1 \times 10^{-48}\ kg.m^2$ & \option{B} & $2.8 \times 10^{-47}\ kg.m^2$ \\
            \option{C} & $2.8 \times 10^{-48}\ kg.m^2$  & \option{D} & $1.1 \times 10^{-42}\ kg.m^2$ \\
        }
    \end{entry}

    \begin{entry}
        \question{If we take the nuclear spin $ \vec{I}$ into account, the total
        angular momentum is $ \vec{F} = \vec{L} + \vec{S} + \vec{I}$, where $ \vec{L}$ and $ \vec{S}$ are the orbital and spin angular momenta of the electron. The Hamiltonian of the hydrogen atom is corrected by the additional interaction $ \lambda \vec{I} \cdot (\vec{L} + \vec{S})$, where $ \lambda > 0$ is a constant. The total angular momentum quantum number F of the p-orbital state with the lowest energy is \\ (November 19, 2020)}
        \options{
            \option{A} & 0   & \option{B} & 1\\
            \option{C} & 1/2 & \option{D} & 3/2\\
        }
    \end{entry}

    \begin{entry}
        \question{The cavity ofa He-Ne laser emitting at 632.8 nm, consists of two mirrors separated by a distance of 35 cm. If the oscillations in the laser cavity occur at frequencies within the gain bandwidth of 1.3 GHz, the number of longitudinal modes allowed in the cavity is \\ (June 16, 2019)}
        \options{
            \option{A} & 1 & \option{B} & 2\\
            \option{B} & 3 & \option{D} & 4\\
        }
    \end{entry}

    \begin{entry}
        \question{The energy levels corresponding to the rotational motion of a molecule are $E_J = BJ(J + 1)\ cm^{-1}$ where $J = 0, 1, 2, \ldots$ and B is a constant. Pure rotational Raman transitions follow the selection rule $ \Delta J = 0, \pm 2$. When the molecule is irradiated, the separation between the closest Stokes and anti-Stokes lines (in $cm^{-1}$) is \\ (June 16, 2019)}
        \options{
            \option{A} & 6B & \option{B} & 12B\\
            \option{C} & 4B & \option{D} & 8B\\
        }
    \end{entry}

    \begin{entry}
        \question{A bound electron and hole pair interacting via Coulomb interaction in a semiconductor is called an exciton. The effective masses of an electron and a hole are about $0.1\ m_e$ and $0.5\ m_e$ respectively, where $m_e$ is the rest mass of the electron. The dielectric constant of the semiconductor 10. Assuming that the energy levels of the excitons are hydrogenlike, the binding energy of an exciton (in units of the Rydberg constant) is closest to \\ (June 16, 2019)}
        \options{
            \option{A} & $2 \times 10^{-3}$ & \option{B} & $2 \times 10^{-4}$ \\
            \option{C} & $8 \times 10^{-4}$ & \option{D} & $3 \times 10^{-3}$ \\
        }
    \end{entry}

    \begin{entry}
        \question{In a spectrum resulting from Raman scattering, let $I_R$ denote the intensity of Rayleigh scattering and $I_S$ and $I_{AS}$ denote the most intense Stokes line and the most intense anti-Stokes line, respectively. The correct order of these Intensities is \\ (December 15, 2019)}
        \options{
            \option{A} & $I_S > I_R > I_{AS}$ \\
            \option{B} & $I_R > I_S > I_{AS}$ \\
            \option{C} & $I_{AS} > I_R > I_S$ \\
            \option{D} & $I_R > I_{AS} > I_S$ \\
        }
    \end{entry}

    \begin{entry}
        \question{The outermost shell of an atom of an element is $3d^3$. The spectral symbol for the ground state is \\ (December 15, 2019)}
        \options{
            \option{A} & $^4F_{3/2}$ & \option{B} & $^4F_{9/2}$ \\
            \option{C} & $^4D_{7/2}$ & \option{D} & $^4D_{1/2}$ \\
        }
    \end{entry}

    \begin{entry}
        \question{The mean kinetic energy per atom in a sodium vapour lamp is 0.33 eV. Given that the mass of sodium atom is approximately $22.5 x 10^9\ eV$, the ratio of the Doppler width of an optical line to its central frequency is \\ (December 15, 2019)}
        \options{
            \option{A} & $7 \times 10^{-7}$ & \option{B} & $6 \times 10^{-6}$ \\
            \option{C} & $5 \times 10^{-5}$ & \option{D} & $4 \times 10^{-4}$ \\
        }
    \end{entry}

    \begin{entry}
        \question{A negative muon, which has a mass nearly 200 times that of an electron, replaces an electron in a Li atom. The lowest ionization energy for the muonic Li atom is approximately \\ (December 15, 2019)}
        \options{
            \option{A} & the same as that of He \\
            \option{B} & the same as that of normal Li \\
            \option{C} & 200 times larger than that of normal Li \\
            \option{D} & the same as that of normal Be \\
        }
    \end{entry}

    \begin{entry}
        \question{If the coefficient of stimulated emission for a particular transition is $2.1 \times 10^{19}\ m^3 W^{-1}s^{-3}$ and the emitted photon is at wavelength $3000 \textup{~\AA}$, then the lifetime of the excited state is approximately \\ (June 18, 2017)}
        \options{
            \option{A} & 20 ns & \option{B} & 40 ns \\
            \option{C} & 80 ns & \option{D} & 100 ns \\
        }
    \end{entry}

    \begin{entry}
        \question{An atomic spectral line is observed to split into nine components due to Zeeman shift. If the upper state of the atom is $^3D_2$ then the lower state will be \\ (June 18, 2017)}
        \options{
            \option{A} & $^3F_2$ & \option{B} & $^3F_1$ \\
            \option{C} & $^3P_1$ & \option{D} & $^3P_2$ \\
        }
    \end{entry}

    \begin{entry}
        \question{If the fine structure splitting between the $^2P_{3/2}$ and $^2P_{1/2}$ levels in the hydrogen atom is $0.4\ cm^{-1}$, the corresponding splitting in $Li^{+2}$ will approximately be \\ (December 17, 2017)}
        \options{
            \option{A} & $ 1.2\ cm^{-1}$ & \option{B} & $10.8\ cm^{-1}$ \\
            \option{C} & $32.4\ cm^{-1}$ & \option{D} & $36.8\ cm^{-1}$ \\
        }
    \end{entry}

    \begin{entry}
        \question{The separations between the adjacent levels of a normal multiplet are found to be $22\ cm^{-1}$ and $33\ cm^{-1}$. Assume that the multiplet is described well by the L-S coupling scheme and the Lande's interval rule, namely $E(J) - E(J - 1) = AJ$, where A is a constant. The term notations for this multiplet is \\ (December 17, 2017)}
        \options{
            \option{A} & $^3P_{0,1,2}$ & \option{B} & $^3F_{2,3,4}$ \\
            \option{C} & $^3G_{3,4,5}$ & \option{D} & $^3D_{1,2,3}$ \\
        }
    \end{entry}

    \begin{entry}
        \question{The Zeeman shift of the energy of a state with quantum numbers $L$, $S$, $J$ and $m_j$ is
        $$ H_z = \frac{m_J \mu_B B}{J(J+1)} (\langle \vec{L} \cdot \vec{J}\rangle + g_s \langle \vec{S} \cdot \vec{J} \rangle) $$
        where $B$ is the applied magnetic field, $g_s$ is the g-factor for the spin and $\mu_B/h = 1.4\ \text{MHz.G}^{-1}$, where $h$ is the Planck constant. The approximate frequency shift of the $S=0$, $L=1$, and $m_J=1$, at a magnetic field of 1G, is \\ (December 17, 2017)}
        \options{
            \option{A} & 10 MHz & \option{B} & 1.4 MHz \\
            \option{C} &  5 MHz & \option{D} & 2.8 MHz \\
        }
    \end{entry}

\end{document}
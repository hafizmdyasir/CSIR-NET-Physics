% amp.tex
% Created by Mohammad Yasir.


\documentclass[12pt]{report}

\usepackage{parskip, amsmath, amssymb, titling}
\usepackage[a4paper, top=1.00in, bottom=1.00in, left=1.00in, right=1.00in]{geometry}

\usepackage[T1]{fontenc}
\usepackage{mlmodern}

\usepackage{../styles}
\pagestyle{plain}

\title{Atomic and Molecular Physics}
\author{Mohammad Yasir}
\date{\today}

\begin{document}
\begin{Huge}
    \thetitle
\end{Huge}\\
\begin{small}
    \textbf{Compiled by :} Mohammad Yasir\\
    \textbf{Updated \hspace{1.78em}:} \thedate
\end{small}
\hrule

% PART B
    \section*{Part B - Advanced}
    \textbf{Quick Note:} Atomic and Molecular Physics is not included in the Core Part A syllabus.


    \begin{entry}
        \question{Diffuse hydrogen gas within a galaxy may be assumed to follow a Maxwell distribution at temperature $10^6\ K$, while the temperature appropriate for the H gas in the inter-galactic space, following the same distribution, may be taken to be $10^4\ K$. The ratio of thermal broadening of the Lyman-$\alpha$ line from the H-atoms within the galaxy to that from the inter-galactic space is closest to. \\ \hfill (February 15, 2022)}
        \options{
            \option{A} & 100 & \option{B} & 1/100\\
            \option{C} & 10  & \option{D} & 1/10\\
        }
    \end{entry}

    \begin{entry}
        \question{The absorption lines arising from pure rotational effects of HCl are observed at $83.03\ cm^{-1}$, $103.73\ cm^{-1}$, $124.30\ cm^{-1}$, $145.03\ cm^{-1}$ and $165.51\ cm^{-1}$. The moment of inertia of the HCl molecule is
        (take $\frac{\hbar}{2\pi c} = 5.6 \times 10^{-44} kg.m$)\\ \hfill (November 19, 2020)}
        \options{
            \option{A} & $1.1 \times 10^{-48}\ kg.m^2$ & \option{B} & $2.8 \times 10^{-47}\ kg.m^2$\\
            \option{C} & $2.8 \times 10^{-48}\ kg.m^2$  & \option{D} & $1.1 \times 10^{-42}\ kg.m^2$\\
        }
    \end{entry}

    \begin{entry}
        \question{If we take the nuclear spin $\vec{I}$ into account, the total
        angular momentum is $\vec{F} = \vec{L} + \vec{S} + \vec{I}$, where $\vec{L}$ and $\vec{S}$ are the orbital and spin angular momenta of the electron. The Hamiltonian of the hydrogen atom is corrected by the additional interaction $\lambda \vec{I} \cdot (\vec{L} + \vec{S})$, where $\lambda > 0$ is a constant. The total angular momentum quantum number F of the p-orbital state with the lowest energy is \\ \hfill (November 19, 2020)}
        \options{
            \option{A} & 0   & \option{B} & 1\\
            \option{C} & 1/2 & \option{D} & 3/2\\
        }
    \end{entry}

\end{document}
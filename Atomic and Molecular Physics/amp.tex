% amp.tex
% Created by Mohammad Yasir.


\documentclass[12pt]{report}

\usepackage{parskip, amsmath, amssymb, titling}
\usepackage[a4paper, top=1.00in, bottom=1.00in, left=1.00in, right=1.00in]{geometry}

\usepackage[T1]{fontenc}
\usepackage{mlmodern}

\usepackage{../styles}
\pagestyle{plain}

\title{Atomic and Molecular Physics}
\author{Mohammad Yasir}
\date{\today}

\begin{document}
\begin{Huge}
    \thetitle
\end{Huge}\\
\begin{small}
    \textbf{Compiled by :} Mohammad Yasir\\
    \textbf{Updated \hspace{1.78em}:} \thedate
\end{small}
\hrule

% PART B
    \section*{Part B - Advanced}
    \textbf{Quick Note:} Atomic and Molecular Physics is not included in the Core Part A syllabus.


    \begin{entry}
        \question{Diffuse hydrogen gas within a galaxy may be assumed to follow a Maxwell distribution at temperature $10^6\ K$, while the temperature appropriate for the H gas in the inter-galactic space, following the same distribution, may be taken to be $10^4\ K$. The ratio of thermal broadening of the Lyman-$\alpha$ line from the H-atoms within the galaxy to that from the inter-galactic space is closest to. \\ \hfill (February 15, 2022)}
        \options{
            \option{A} & 100 & \option{B} & 1/100\\
            \option{C} & 10  & \option{D} & 1/10\\
        }
    \end{entry}

    \begin{entry}
        \question{The absorption lines arising from pure rotational effects of HCl are observed at $83.03\ cm^{-1}$, $103.73\ cm^{-1}$, $124.30\ cm^{-1}$, $145.03\ cm^{-1}$ and $165.51\ cm^{-1}$. The moment of inertia of the HCl molecule is
        (take $\frac{\hbar}{2\pi c} = 5.6 \times 10^{-44} kg.m$)\\ \hfill (November 19, 2020)}
        \options{
            \option{A} & $1.1 \times 10^{-48}\ kg.m^2$ & \option{B} & $2.8 \times 10^{-47}\ kg.m^2$\\
            \option{C} & $2.8 \times 10^{-48}\ kg.m^2$  & \option{D} & $1.1 \times 10^{-42}\ kg.m^2$\\
        }
    \end{entry}

    \begin{entry}
        \question{If we take the nuclear spin $\vec{I}$ into account, the total
        angular momentum is $\vec{F} = \vec{L} + \vec{S} + \vec{I}$, where $\vec{L}$ and $\vec{S}$ are the orbital and spin angular momenta of the electron. The Hamiltonian of the hydrogen atom is corrected by the additional interaction $\lambda \vec{I} \cdot (\vec{L} + \vec{S})$, where $\lambda > 0$ is a constant. The total angular momentum quantum number F of the p-orbital state with the lowest energy is \\ \hfill (November 19, 2020)}
        \options{
            \option{A} & 0   & \option{B} & 1\\
            \option{C} & 1/2 & \option{D} & 3/2\\
        }
    \end{entry}

    \begin{entry}
        \question{The cavity ofa He-Ne laser emitting at 632.8 nm, consists of two mirrors separated by a distance of 35 cm. If the oscillations in the laser cavity occur at frequencies within the gain bandwidth of 1.3 GHz, the number of longitudinal modes allowed in the cavity is \\ \hfill (June 16, 2019)}
        \options{
            \option{A} & 1 & \option{B} & 2\\
            \option{B} & 3 & \option{D} & 4\\
        }
    \end{entry}

    \begin{entry}
        \question{The energy levels corresponding to the rotational motion of a molecule are $E_J = BJ(J + 1)\ cm^{-1}$ where $J = 0, 1, 2, \ldots$ and B is a constant. Pure rotational Raman transitions follow the selection rule $\Delta J = 0, \pm 2$. When the molecule is irradiated, the separation between the closest Stokes and anti-Stokes lines (in $cm^{-1}$) is \\ \hfill (June 16, 2019)}
        \options{
            \option{A} & 6B & \option{B} & 12B\\
            \option{C} & 4B & \option{D} & 8B\\
        }
    \end{entry}

    \begin{entry}
        \question{A bound electron and hole pair interacting via Coulomb interaction in a semiconductor is called an exciton. The effective masses of an electron and a hole are about $0.1\ m_e$ and $0.5\ m_e$ respectively, where $m_e$ is the rest mass of the electron. The dielectric constant of the semiconductor 10. Assuming that the energy levels of the excitons are hydrogenlike, the binding energy of an exciton (in units of the Rydberg constant) is closest to \\ \hfill (June 16, 2019)}
        \options{
            \option{A} & $2 \times 10^{-3}$ & \option{B} & $2 \times 10^{-4}$\\
            \option{C} & $8 \times 10^{-4}$ & \option{D} & $3 \times 10^{-3}$ \\
        }
    \end{entry}

\end{document}
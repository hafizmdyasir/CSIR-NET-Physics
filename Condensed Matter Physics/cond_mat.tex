% cond_mat.tex
% Created by Mohammad Yasir.


\documentclass[12pt]{report}

\usepackage{parskip, amsmath, amssymb, titling}
\usepackage[a4paper, top=1.00in, bottom=1.00in, left=1.00in, right=1.00in]{geometry}

\usepackage[T1]{fontenc}
\usepackage{mlmodern}

\usepackage{../styles}
\pagestyle{plain}

\title{Condensed Matter Physics}
\author{Mohammad Yasir}
\date{\today}

\begin{document}
\begin{Huge}
    \thetitle
\end{Huge}\\
\begin{small}
    \textbf{Compiled by :} Mohammad Yasir\\
    \textbf{Updated \hspace{1.78em}:} \thedate
\end{small}
\hrule

% PART B
\section*{Part B - Advanced}
\textbf{Quick Note:} Condensed Matter Physics is not included in the Core Part A syllabus.

\begin{entry}
    \question{Lead is superconducting below 7 K and has a critical magnetic field $800 \times 10^{-4}$ tesla close to 0 K. At 2 K the critical current that flows through a long lead wire of radius 5 mm is closest to\\(February 15, 2022)}

    \options{
        \option{A}{1760 A} \\ \option{B}{1670 A}\\
        \option{C}{1950 A} \\ \option{D}{1840 A}\\
    }
\end{entry}


\begin{entry}
    \question{A lattice is defined by the unit vectors $\vec{a_1} = a \hat{i}$, $\vec{a_2} = -\frac{a}{2} \hat{i} + \frac{a\sqrt{3}}{2} \hat{j}$, and $\hat{a_3} = a \hat{k}$, where $a > 0$ is a constant. The spacing between the (100) planes of the lattice is\\(November 19, 2020)}
    \options{
        \option{A}{$\sqrt{3}a/2$} \\ \option{B}{$a/2$} \\
        \option{C}{$a$} \\ \option{D}{$\sqrt{2}a$}\\
    }
\end{entry}

\begin{entry}
    \question{A tight binding model of electrons in one dimension has the dispersion relation $\varepsilon(k) = -2t (1 - \cos{ka})$, where $t > 0$, $a$ is the lattice constant and $\frac{-\pi}{a} < k < \frac{\pi}{a}$. Which of the following figures best represents the density of states over the range $\frac{\pi}{2a} \leq k < \frac{\pi}{a}$?\\(November 19, 2020)}

    \options{
        \option[images/Tight Binding Nov 19 A.png]{A}{} \\
        \option[images/Tight Binding Nov 19 B.png]{B}{} \\
        \option[images/Tight Binding Nov 19 C.png]{C}{} \\
        \option[images/Tight Binding Nov 19 D.png]{D}{} \\
    }
\end{entry}

\begin{figure*}
    \includegraphics[width=0.35\textwidth]{images/BBR Nov 19.png}
\end{figure*}
\begin{entry}
    \question{The energy density $I$ of a black body radiation at temperature $T$ is given the Planck's distribution function $$I(\nu, T) = \frac{8\pi\nu^2}{c^3} \frac{h\nu}{(e^{\frac{h\nu}{k_BT}} - 1)}$$where $\nu$ is the frequency. The frequency $I(\nu, T)$ for two different temperatures $T_1$ and $T_2$ are shown above:
        If the two curves coincide when $I(\nu, T)\nu^a$ is plotted against $\nu^b/T$, then the values of $a$ and $b$ are, respectively,\\
        (November 19, 2020)}
    \options{
        \option{A}{2 and 1}\\
        \option{B}{-2 and 2}\\
        \option{C}{3 and -1}\\
        \option{D}{-3 and 1}\\
    }
\end{entry}

\begin{entry}
    \question{For an ideal gas consisting of $N$ distinguishable particles in a volume $V$, the probability of finding exactly 2 particles in a volume $\delta V << V$, in the limit $N,\ V \rightarrow \infty$, is\\
        (November 19, 2020)}
    \options{
        \option{A}{$2N\delta V/V$}\\
        \option{B}{($N\delta V/V)^2$}\\
        \option{C}{$\frac{(N\delta V)^2}{2V^2} e^{-N\delta V/V}$}\\
        \option{D}{$\left(\frac{\delta V}{V}\right)^2 e^{-N\delta V/V}$}\\
    }
\end{entry}


\end{document}
